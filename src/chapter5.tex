\section{Économie internationale et globalisation} % (fold)
\label{sec:economie_internationale_et_globalisation}

Il est important de savoir que depuis la fin de la seconde guerre mondiale,
les échanges internationaux ainsi que la circulation des capitaux se
sont beaucoup développés.
Les évolutions des taux de change ont des effets sur la compétitivité
des nations et donc sur l'économie réelle.
Cette mondialisation est possible notamment grâce à la diminution
des coûts des transports (feroviaire et maritime)
ainsi que la chûte régulière des droits de douane.
On s'attechera dans cette section à décrire deux points
\begin{enumerate}
  \item Les théories du \emph{commerce international} qui étudient les flux réels de biens 
  et de services d'une part,
  \item et les théories de la \emph{finance internationale} qui analysent les flux 
  financiers d'autre part.
\end{enumerate}
On discutera également les impacts, qu'ils soient positifs ou négatifs,
de ces théories sur la mondialisation.

\subsection{Commerce international} % (fold)
\label{sub:commerce_international}
Il paraît interessant de noter qu'en moyenne, les petits pays font plus 
de commerce avec les autres pays, que les grands.
Ceci s'expliquent par une production interne moins diversifiée
et à moins grande échelle.

À titre d'exemple, on notera que les accords au sein de l'Union Européenne
sur le libre-échange ont permis de porter le commerce entre pays européens au tiers
du commerce mondial.


\subsubsection{Régulation du commerce mondial} % (fold)
\label{ssub:regulation_du_commerce_mondial}
On va maintenant brièvement décrire les différents types d'accords internationaux
qui nous ont amenés à la situation actuelle.

Les partenaires principaux du commerce international sont contraints par le
\textsc{gatt} (General Agreements on Tarifs and Trade) par trois principes,
à savoir
\begin{enumerate}
  \item Le principe de \emph{non-discrimination} dicte que tout avantage tarifaire accordé
  à un membre doit être étendu à l'ensemble des membres,
  \item le principe de \emph{réciprocité} dicte qu'on ne peut bénéficier des concessions
  de ses partenaires sans en accorder soi-même,
  \item et le principe de \emph{transparence} qui dicte que les barrières commerciales de
  types quotas doivent être converties en droits de douane.
\end{enumerate}
% subsubsection regulation_du_commerce_mondial (end)
On retrouve ensuite deux \textbf{interdictions} qui viennent compléter ces trois principes
\begin{enumerate}
  \item interdiction de \emph{dumping}, qui correspond à  exporter une marchandise 
  à un prix inférieur à celui pratiqué dans le pays d'origine,
  \item et l'interdiction des \emph{subventions} lorsqu'elles maintiennent des prix
  artificiellement faibles.
\end{enumerate}

% subsection commerce_international (end)

\subsubsection{Théories du commerce international} % (fold)
\label{ssub:theories_du_commerce_international}
On retrouve les \emph{avantages absolus de Smith} ainsi que
les \emph{avantages comparatifs de Ricardo}.
Cependant, ces deux-ci ne permettent pas d'expliquer tous les phénomènes
actuels observés.
Par exemple, lorsqu'on s'intéresse à l'existence de rendements d'échelle
croissant qui entraînent la concentration de la production d'un bien 
dans un seul pays, il faut faire appel au modèle de \emph{concurrence imparfaite}.



% subsubsection theories_du_commerce_international (end)

\subsection{Finance internationale} % (fold)
\label{sub:finance_internationale}

% subsection finance_internationale (end)

% section economie_internationale_et_globalisation (end)